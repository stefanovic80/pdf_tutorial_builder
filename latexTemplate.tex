
%\documentclass[12pt]{report}
%\documentclass[12pt]{extreport}
\documentclass[17pt]{extarticle}
%\documentclass{memoir}

\usepackage{graphicx}
\usepackage{setspace}
\usepackage{amsmath,amssymb}
\usepackage{IEEEtrantools}
\usepackage{cancel}
\usepackage[font=small,labelfont=bf]{caption}


\usepackage{verbatim}

\usepackage[T1]{fontenc}
\usepackage[utf8]{inputenc}
\usepackage[italian]{babel}

\usepackage{imakeidx}
\usepackage{hyperref}
\makeindex


\usepackage{geometry}
 \geometry{
 a4paper,
 total={170mm,264mm},
 left=2mm,
 top=10mm,
 }



\newcommand{\pict}[1]{
\begin{figure}[h!]		
	\centering
   	\includegraphics[width=8.0in]{pictures/picture_#1.png}
  	\caption{#1}
   	\label{fig:LibreOfficeCalc#1}
\end{figure}
}



\begin{document}

\begin{flushright}
{\bf \today}
\end{flushright}

\tableofcontents

\clearpage


Abbiamo visto analiticamente\footnote{Cioè facendo i calcoli} che, data una equazione lineare in due incognite $x$ e $y$, una sua possibile soluzione è una \emph{coppia ordinata} di numeri. 


Ad esempio, data l'equazione 

\begin{equation}\label{eq:eqLineare}
	9y - 7x - 3 = 0
\end{equation}

attribuendo a $x$ il valore $0$, si ottiene per $y$ il valore $\frac{1}{3}$, attribuendo a $x$ il valore $1$, si ottiene per $y$ il valore $\frac{10}{9}$ e così via, come nella tabella che segue


\begin{center}
\begin{tabular}%\label{tab:eqLineare}
	{ |c|c| c| }
 \hline
 {\bf x} &  {\bf y} &  $y$ approssimato \\ 
 \hline
 0 & $\frac{1}{3}$  &  $0,33$ \\
 1 & $\frac{10}{9}$ &  $0,11$ \\ 
 2 & $\frac{17}{9}$ &  $1,89$ \\ 
 3 & $\frac{24}{9}$ &  $2,67$ \\ 
 4 & $\frac{31}{9}$ &  $3,44$ \\ 
 5 & $\frac{38}{9}$ &  $4,22$ \\ 
$\dots$ & $\dots$	& $\dots$\\	
 \hline

%\caption{Tabella delle possibile coppie ordinate di valori $x$ e $y$ soluzioni dell'equazione lineare in due incognite.}
\end{tabular}
\end{center}


%essendo \emph{infiniti} i differenti valori che posso attribuire a $x$, infiniti sono anche i possibili valori di $y$. 

Abbiamo anche visto, adoperando la carta millimetrata, che le infinite coppie ordinate che risolvono una equazione lineare come la eq. \ref{eq:eqLineare} \emph{giacciono tutte e solo} su una ben precisa retta (fig. \ref{fig:LibreOfficeCalc000}). 



\pict{000}

% Le coppie ordinate della tabella precedente, soddisfano l'equazione lineare $9y -7x - 3 = 0$ e quindi corrispondono a punti sul piano cartesiano che stanno tutti su una stessa retta.

Sempre adoperando la carta millimetrata, abbiamo visto che mettendo a sistema due differenti equazioni lineari, la coppia ordinata risultato del sistema è, graficamente, il punto di intersezione delle rispettive rette. In questa sede andiamo a rivedere tutto questo facendo uso del \emph{foglio di calcolo elettronico} e, in particolare, di LibreOffice Calc.

\newpage

\section{Rappresentazione grafica di una retta su piano cartesiano con \emph{foglio di calcolo elettronico} }


%Inoltre, il risultato di un sistema due equazioni lineari e due incognite, quando ammette una unica soluzione\footnote{le soluzioni possono essere \emph{infinite} quando il sistema è indeterminato o nessuna, quando il sistema è \emph{impossibile}.}, è una \emph{coppia ordinata} di numeri che corrispondono alle coordinate del punto di intersezione tra le due rette associate a ciascuna delle due equazioni.

%Data una equazione lineare in due variabili $x$ e $y$, come ad esempio

%usiamo il foglio di calcolo elettronico e, in particolare, \emph{LibreOffice Calc}, per calcolare il valore delle coppie ordinate che soddisfano una equazione

%sono infinite
Prendiamo in considerazione il sistema due equazioni lineari e due incognite:

\begin{equation}\label{eq:sistema}
  	\begin{cases}
		9y - 2x + 12 = 0\\
		9y + 7x + 3 = 0		
	\end{cases}
\end{equation}

Facendo uso di LibreOffice Calc, per ciascuna delle equazioni di questo sistema, andiamo a calcolare una tabella, come quella del paragrafo precedente.

Per fare ciò, conviene che per ciascuna delle due equazioni del sistema \ref{eq:sistema}, esplicitiamo la $y$ in funzione della $x$


\begin{equation}\label{eq:sistemaEspl}
  	\begin{cases}
		y = \frac{2}{9}x - \frac{4}{3}\\
		y = -\frac{7}{9}x - \frac{1}{3}
	\end{cases}
\end{equation}


Passiamo ora a LibreOffice Calc (fig. \ref{fig:LibreOfficeCalc000})
%\begin{figure}[h!]		
	\centering
   	\includegraphics[width=8.0in]{pictures/picture_003.png}
  	\caption{LibreOffice Calc}
   	\label{fig:LibreOfficeCalc003}
\end{figure}

\pict{003}

\newpage

\section{Scrittura nelle caselle di testo, numeri e formule matematiche}

Usiamo la colonna A i valori delle $x$ che andiamo a scegliere, la colonna B per i valori di $y$ della prima equazione ($y1$), e la colonna C per i valori di $y2$ della seconda equazione ($y2$). Per ciascuna colonna, la prima riga la usiamo per dare un "titolo alla colonna stessa.

Usiamo la seconda colonna per il primo valore di $x$, che fissiamo pari a $-10$, il primo valore di $y1$ e il primo valore di $y2$, che invece devono essere calcolati a partire dal sistema \ref{eq:sistema}; come in figura \ref{fig:LibreOfficeCalc004}

%\begin{figure}[h!]		
	\centering
   	\includegraphics[width=8.0in]{pictures/picture_004.png}
  	\caption{LibreOffice Calc}
   	\label{fig:LibreOfficeCalc004}
\end{figure}
\pict{004}

Per calcolare il valore di $y1$ nella casella $B2$ dobbiamo inizializzare la casella con il carattere "=" e, successivamente, mettere l'espressione matematica a secondo membro della prima equazione del sistema \ref{eq:sistemaEspl}.

%\begin{figure}[h!]		
	\centering
   	\includegraphics[width=8.0in]{pictures/picture_005.png}
  	\caption{LibreOffice Calc}
   	\label{fig:LibreOfficeCalc005}
\end{figure}
\pict{005}

Allo stesso modo, nella casella $C2$ immattiamo il secondo membro della seconda equazione del sistema \ref{eq:sistemaEspl}

%\begin{figure}[h!]		
	\centering
   	\includegraphics[width=8.0in]{pictures/picture_006.png}
  	\caption{LibreOffice Calc}
   	\label{fig:LibreOfficeCalc006}
\end{figure}
\pict{006}


Il risultato che dovrebbe apparire è riportato nella figura \ref{fig:LibreOfficeCalc007}


%\begin{figure}[h!]		
	\centering
   	\includegraphics[width=8.0in]{pictures/picture_007.png}
  	\caption{LibreOffice Calc}
   	\label{fig:LibreOfficeCalc007}
\end{figure}
\pict{007}


\newpage
\section{Copiatura delle formule matematiche su più caselle della medesima colonna}

Scriviamo ora l'espressione algebrica "$=a2 + 0,01$" nella casella A3

%\begin{figure}[h!]		
	\centering
   	\includegraphics[width=8.0in]{pictures/picture_008.png}
  	\caption{LibreOffice Calc}
   	\label{fig:LibreOfficeCalc008}
\end{figure}
\pict{008}


Poi selezioniamo le caselle $B2$ e $C2$ (figura \ref{fig:LibreOfficeCalc009}). Cliccando sul quadratino in basso a destra e tenendo premuto fino a scorrere alle due caselle di sotto, dovremmo ottenere il risultato della figura \ref{fig:LibreOfficeCalc010}. E cioè le formule matematiche scritte in $B2$ e in $C2$ in funzione della variabile $x$ presente nella casella $A2$ adesso sono state copiate in $B3$ e $C3$ rispettivamente e entrambe sono in funzione della variabile $x$ presente nella cadella $A3$.

%\begin{figure}[h!]		
	\centering
   	\includegraphics[width=8.0in]{pictures/picture_009.png}
  	\caption{LibreOffice Calc}
   	\label{fig:LibreOfficeCalc009}
\end{figure}
\pict{009}

%\begin{figure}[h!]		
	\centering
   	\includegraphics[width=8.0in]{pictures/picture_010.png}
  	\caption{LibreOffice Calc}
   	\label{fig:LibreOfficeCalc010}
\end{figure}
\pict{010}

Selezioniamo ora le caselle $A3$, $B3$ e $C3$, come in figura \ref{fig:LibreOfficeCalc011}

%\begin{figure}[h!]		
	\centering
   	\includegraphics[width=8.0in]{pictures/picture_011.png}
  	\caption{LibreOffice Calc}
   	\label{fig:LibreOfficeCalc011}
\end{figure}
\pict{011}

E ripetiamo il passaggio precedente, copiando le rispettive formule matematiche per un numero di righe sufficientemente grande da arrivare fino a $x = 10$ (figura \ref{fig:LibreOfficeCalc012})

%\begin{figure}[h!]		
	\centering
   	\includegraphics[width=8.0in]{pictures/picture_012.png}
  	\caption{LibreOffice Calc}
   	\label{fig:LibreOfficeCalc012}
\end{figure}
\pict{012}

In questo modo LibreOffice Calc ha eseguito ben 200 calcoli, e con estrema rapidità.

\newpage

\section{Eseguire il grafico}


Eseguire i seguenti step per riportare le rette associate alle due equazioni del sistema in un unico grafico:

\begin{enumerate}
	\item tornare in alto sul foglio elettronico e selezionare le tre colonne $A$, $B$ e $C$, come in figura 
	
	%\begin{figure}[h!]		
	\centering
   	\includegraphics[width=8.0in]{pictures/picture_013.png}
  	\caption{LibreOffice Calc}
   	\label{fig:LibreOfficeCalc013}
\end{figure}
	\pict{013}	
		
	\item Nel menù in alto selezionare "insert -> chart", figura \ref{fig:LibreOfficeCalc014}. Attendere un po' se il computer non è molto prestante.
	
	%\begin{figure}[h!]		
	\centering
   	\includegraphics[width=8.0in]{pictures/picture_014.png}
  	\caption{LibreOffice Calc}
   	\label{fig:LibreOfficeCalc014}
\end{figure}
	\pict{014}	
	
	\item Selezionare "XY (Scatter)" nella finestra che si apre e, successivamente, "Lines Only" nel riquadro a fianco, e infine premere su "finish".
	
	%\begin{figure}[h!]		
	\centering
   	\includegraphics[width=8.0in]{pictures/picture_015.png}
  	\caption{LibreOffice Calc}
   	\label{fig:LibreOfficeCalc015}
\end{figure}
	\pict{015}
	
	\item Il grafico della figura \ref{fig:LibreOfficeCalc016} dovrebbe apparire

	%\begin{figure}[h!]		
	\centering
   	\includegraphics[width=8.0in]{pictures/picture_016.png}
  	\caption{LibreOffice Calc}
   	\label{fig:LibreOfficeCalc016}
\end{figure}
	\pict{016}

\end{enumerate}
Il punto di intersezione delle due rette dovrebbe avere, appunto, come coordinate la risoluzione del sistema eq. \ref{eq:sistema} o, parimenti, del sistema \ref{eq:sistemaEspl}. Il problema, però, è che questo grafico \emph{non} ha una griglia abbastanza fitta e l'analisi, fino a questo punto, può essere fatta solo in linea di massima. Possiamo allargare il grafico andando a cliccare sui quadratini agli angoli del riquadro che lo delimita (figura \ref{fig:LibreOfficeCalc017}


%\begin{figure}[h!]		
	\centering
   	\includegraphics[width=8.0in]{pictures/picture_017.png}
  	\caption{LibreOffice Calc}
   	\label{fig:LibreOfficeCalc017}
\end{figure}
\pict{017}


Andiamo a vedere come si possono impostare i parametri del grafico, in modo da accertarci che il punto di intersezione sia effettivamente quello cercato.

\newpage

\section{Impostazione dei parametri del grafico}

In primo luogo notiamo che i valori delle $x$ vanno da $-15$ a $+15$, mentre a noi basta che vadano da $-10$ a $+10$. Nella figura \ref{fig:LibreOfficeCalc017} facciamo doppio click su uno dei numeri dell'asse delle $x$, ad esempio il numero $-5$ e si apre la finestra di figura \ref{fig:LibreOfficeCalc018}

%\begin{figure}[h!]		
	\centering
   	\includegraphics[width=8.0in]{pictures/picture_018.png}
  	\caption{LibreOffice Calc}
   	\label{fig:LibreOfficeCalc018}
\end{figure}
\pict{018}

Leviamo la spunta su "automatic" su tutte e quattro le voci, cambiamo "Minimum" e "Maximum" in $-10$ e $+10$, cambiamo anche "Major interval" e "Minor interval count" in modo da impostare una griglia più vicina a quella di una carta millimetrata, come in figura \ref{fig:LibreOfficeCalc019}


%\begin{figure}[h!]		
	\centering
   	\includegraphics[width=8.0in]{pictures/picture_019.png}
  	\caption{LibreOffice Calc}
   	\label{fig:LibreOfficeCalc019}
\end{figure}
\pict{019}

E il risultato dovrebbe essere quello di figura \ref{fig:LibreOfficeCalc020}

%\begin{figure}[h!]		
	\centering
   	\includegraphics[width=8.0in]{pictures/picture_020.png}
  	\caption{LibreOffice Calc}
   	\label{fig:LibreOfficeCalc020}
\end{figure}
\pict{020}

Allo stesso modo, cliccando su un numero dell'asse delle $y$, ad esempio il numero $2$, si possono configurare i valori massimo e minimo di $y$, nonchè i valori della griglia

\ref{fig:LibreOfficeCalc021}


%\begin{figure}[h!]		
	\centering
   	\includegraphics[width=8.0in]{pictures/picture_021.png}
  	\caption{LibreOffice Calc}
   	\label{fig:LibreOfficeCalc021}
\end{figure}
\pict{021}


\newpage
\section{Cambiare il font dei numeri sugli assi}

Nella figura \ref{fig:LibreOfficeCalc022} risulta che bisogna cliccare prima su font e poi scegliere il valore desiderato.
%\begin{figure}[h!]		
	\centering
   	\includegraphics[width=8.0in]{pictures/picture_022.png}
  	\caption{LibreOffice Calc}
   	\label{fig:LibreOfficeCalc022}
\end{figure}
\pict{022}


nella figura \ref{fig:LibreOfficeCalc023} il risultato, dopo aver impostato un font pari a 16 \emph{sia} per i numeri dell'asse delle $x$, sia per i numeri dell'asse delle $y$

%\begin{figure}[h!]		
	\centering
   	\includegraphics[width=8.0in]{pictures/picture_023.png}
  	\caption{LibreOffice Calc}
   	\label{fig:LibreOfficeCalc023}
\end{figure}
\pict{023}

\section{Inserire una griglia più fitta}

%\begin{figure}[h!]		
	\centering
   	\includegraphics[width=8.0in]{pictures/picture_024.png}
  	\caption{LibreOffice Calc}
   	\label{fig:LibreOfficeCalc024}
\end{figure}
\pict{024}


%\begin{figure}[h!]		
	\centering
   	\includegraphics[width=8.0in]{pictures/picture_025.png}
  	\caption{LibreOffice Calc}
   	\label{fig:LibreOfficeCalc025}
\end{figure}
\pict{025}


%\begin{figure}[h!]		
	\centering
   	\includegraphics[width=8.0in]{pictures/picture_026.png}
  	\caption{LibreOffice Calc}
   	\label{fig:LibreOfficeCalc026}
\end{figure}
\pict{026}


%\begin{figure}[h!]		
	\centering
   	\includegraphics[width=8.0in]{pictures/picture_027.png}
  	\caption{LibreOffice Calc}
   	\label{fig:LibreOfficeCalc027}
\end{figure}
\pict{027}






\newpage
~\newpage

\section{Infittire la griglia verticale}



%\begin{figure}[h!]		
	\centering
   	\includegraphics[width=8.0in]{pictures/picture_028.png}
  	\caption{LibreOffice Calc}
   	\label{fig:LibreOfficeCalc028}
\end{figure}
\pict{028}

\section{Infittire la griglia orizzontale}

%\begin{figure}[h!]		
	\centering
   	\includegraphics[width=8.0in]{pictures/picture_029.png}
  	\caption{LibreOffice Calc}
   	\label{fig:LibreOfficeCalc029}
\end{figure}
\pict{029}


%\begin{figure}[h!]		
	\centering
   	\includegraphics[width=8.0in]{pictures/picture_030.png}
  	\caption{LibreOffice Calc}
   	\label{fig:LibreOfficeCalc030}
\end{figure}
\pict{030}

\end{document}